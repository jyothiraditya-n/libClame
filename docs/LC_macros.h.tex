% libClame: Command-line Arguments Made Easy
% Copyright (C) 2021-2023 Jyothiraditya Nellakra
%
% This program is free software: you can redistribute it and/or modify it under
% the terms of the GNU General Public License as published by the Free Software
% Foundation, either version 3 of the License, or (at your option) any later 
% version.
%
% This program is distributed in the hope that it will be useful, but WITHOUT
% ANY WARRANTY; without even the implied warranty of MERCHANTABILITY or FITNESS
% FOR A PARTICULAR PURPOSE. See the GNU General Public License for more details.
%
% You should have received a copy of the GNU General Public License along with
% this program. If not, see <https://www.gnu.org/licenses/>.

\section{Annotated Version of the \mintinline{bash}{LC_macros.h} Headerfile.}

You can use these macros in your program by including the \mintinline{bash}{LC_macros.h} header file.\footnotemark This section is an annotated version of this header file. For all of these macros, \monoc{lflag} needs to be a string constant, \monoc{sflag} needs to be a char constant.

\footnotetext{
	The first in each series of macros require libClame Version 1.1 or better, while the remainedr of the macro overloads with flag setting require libClame Version 1.2 or better.
}

\subsection{Flag to Call a Function}
This macro lets you make a flag that calls a helper function you have defined in your code. \monoc{func} needs to be of type \monoc{int func(LC_flag_t *flag)}.\footnotemark

\footnotetext{See Section \ref{sec:callback-function}.}

\begin{minted}{c}
#define LC_MAKE_CALL(lflag, sflag, func) ...
\end{minted}

\subsection{Flag to Set a Boolean to a Given Value}
This macro lets you make a flag to set a boolean to a given value. \monoc{var} needs to be of declared as type \monoc{bool} and \monoc{val} needs to be a boolean constant.

\begin{minted}{c}
#define LC_MAKE_BOOL(lflag, sflag, var, val) ...
#define LC_MAKE_BOOL_F(lflag, sflag, var, val, flag) ...
\end{minted}

\subsection{Flags to Get Config String(s)}
This macro lets you make a flag to set strings. \monoc{var} needs to be declared as type \monoc{char *}, \monoc{arr} needs to be declared as type \monoc{char **}, len needs to be declared as type \monoc{size_t} and \monoc{min(_len)} and \monoc{max(_len)} need to be unsigned integer constants.

\begin{minted}{c}
#define LC_MAKE_STRING(lflag, sflag, var) ...
#define LC_MAKE_STRING_F(lflag, sflag, var, func) ...

#define LC_MAKE_STRING_ARR(lflag, sflag, arr, len) ...
#define LC_MAKE_STRING_ARR_F(lflag, sflag, arr, len) ...

#define LC_MAKE_STRING_ARR_BOUNDED(lflag, sflag, arr, len, min_len, max_len) ...
#define LC_MAKE_STRING_ARR_BOUNDED_F(lflag, sflag, arr, len, min, max, func) ...
\end{minted}

\subsection{Flags to Get a Variable or an Array of Other Types}
This macro lets you make a flag to set any variable or array. For this paragraph, let us denote the type of the data you are treating as \monoc{T}. \monoc{var} needs to be declared as type \monoc{T}, \monoc{arr} needs to be declared as type \monoc{T *}, \monoc{fmt} needs to be a string constant, len needs to be declared as type \monoc{size_t} and \monoc{min(_len)} and \monoc{max(_len)} need to be unsigned integer constants.

\begin{minted}{c}
#define LC_MAKE_VAR(lflag, sflag, var, fmt) ...
#define LC_MAKE_VAR_F(lflag, sflag, var, fmt, func) ...

#define LC_MAKE_ARR(lflag, sflag, arr, fmt, len) ...
#define LC_MAKE_ARR_F(lflag, sflag, arr, fmt, len, func) ...

#define LC_MAKE_ARR_BOUNDED(lflag, sflag, arr, fmt, len, min_len, max_len) ...
#define LC_MAKE_ARR_BOUNDED(lflag, sflag, arr, fmt, len, min, max, func) ...
\end{minted}
