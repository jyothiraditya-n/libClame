% libClame: Command-line Arguments Made Easy
% Copyright (C) 2021-2023 Jyothiraditya Nellakra
%
% This program is free software: you can redistribute it and/or modify it under
% the terms of the GNU General Public License as published by the Free Software
% Foundation, either version 3 of the License, or (at your option) any later 
% version.
%
% This program is distributed in the hope that it will be useful, but WITHOUT
% ANY WARRANTY; without even the implied warranty of MERCHANTABILITY or FITNESS
% FOR A PARTICULAR PURPOSE. See the GNU General Public License for more details.
%
% You should have received a copy of the GNU General Public License along with
% this program. If not, see <https://www.gnu.org/licenses/>.

\section{Annotated Version of the \mintinline{bash}{libClame.hpp} Headerfile.}

You can use this library in your program by including the \mintinline{bash}{libClame.hpp} header file.\footnotemark As such, this reference document provides an annotated version of this header file as a guide to how you should set up your C++ code. It is still worth reading through the documentation for \mintinline{bash}{libClame.h} in Section \ref{sec:libClame.h}, so you can understand the way the C code is set up if you ever need to fall back on it. However, we will also touch upon some parts of that codebase here, where it is useful.

Code and structures that are from C use the convention of starting with the characters \monoc{LC_} while identifiers from C++ land are placed into the namespace \monoc{libClame}.

\footnotetext{
	This requires libClame Version 1.2 or higher.
}

\subsection{Version Information}

\begin{minted}{c}
#define LC_VERSION 1 /* Incremented when backwards compatibility is broken. */
#define LC_SUBVERSION 2 /* Incremented when new features are added. */

#ifdef LC_REQ_VER
#if LC_REQ_VER != LC_VERSION
#error "Incorrect libClame version."
#endif
#endif

#ifdef LC_REQ_SUBVER
#if LC_REQ_SUBVER > LC_SUBVERSION
#error "Incorrect libClame subversion."
#endif
#endif
\end{minted}

\subsection{The \monoc{LC_flag_t} Structure}

Each flag that we process has properties specified by an \monoc{LC_flag_t} struct. A run down of what all of the properties do and mean is written in the comments within the structure definition itself \mintinline{bash}{libClame.h}. For further information see Section \ref{sec:LC_flag_t}.

If you are not digging into the nitty-gritty of things, you can just create a \monoc{std::vector} of \monoc{LC_flag_t} and push flags into it in the main function via the helper functions defined in this header.

\subsubsection{Callback Function}
\label{sec:c++callback-function}

If you have a function that is associated with a certain flag, you can specify it with the type \monoc{callback_t}, taking no arguments and returning nothing either. In order for this function to indicate an error, it must throw an exception, which will be caught by our error handler.

\begin{minted}{c++}
	typedef std::function<void()> callback_t;
\end{minted}

\subsection{Flag to Call a Function}
This macro lets you make a flag that calls a helper function you have defined in your code. \monoc{func} needs to be of type \monoc{libClame::callback_t}.\footnotemark

\footnotetext{See Section \ref{sec:c++callback-function}.}

\begin{minted}{c++}
	extern LC_flag_t make_call(
		std::string lflag, char sflag, callback_t function
	);
\end{minted}

\subsection{Flag to Set a Boolean to a Given Value}
This macro lets you make a flag to set a boolean to a given value. \monoc{var} needs to be of declared as type \monoc{bool} and \monoc{val} needs to be a boolean constant.

\begin{minted}{c++}
	extern LC_flag_t make_bool(
		std::string lflag, char sflag, bool& var, bool val,
		std::optional<callback_t> function
	);
\end{minted}

\subsection{Flags to Get Config String(s)}
This macro lets you make a flag to set either a string or an array of strings.

\begin{minted}{c++}
	extern LC_flag_t make_string(
		std::string lflag, char sflag, std::string& string,
		std::optional<callback_t> function
	);

	typedef std::tuple<size_t, size_t> limits_t;

	extern LC_flag_t make_str_arr(
		std::string lflag, char sflag, std::list<std::string>& strings,
		std::optional<limits_t> limits,
		std::optional<callback_t> function
	);

	extern LC_flag_t make_str_arr(
		std::string lflag, char sflag,
		std::vector<std::string>& strings,
		std::optional<limits_t> limits,
		std::optional<callback_t> function
	);
\end{minted}

\subsubsection{\monoc{limits_t}}
\label{sec:limits_t}

The bounds checking for arrays is implemented through a tuple that contains the minimum and maximum permissible lengths for the array.

\subsection{Calling the Parsing Function}

Once you have set up the flags as described above, you can call the \monoc{libClame::read()} function to process your command line arguments.

\begin{minted}{c++}
extern void read(int argc, char** argv, std::vector<LC_flag_t>& flags);
\end{minted}

\subsection{Returned Values}

\subsubsection{Error Codes}

Upon encountering an error, \monoc{libClame::read()} throws an exception with the following type.

\begin{minted}{c++}
	class exception : std::exception {
	public:
		explicit exception(int error);
		virtual ~exception() noexcept;
		virtual const char *what() const noexcept;
		int error;
	};
\end{minted}

The following are the meanings of value stored in \monoc{error}.

\begin{minted}{c++}
#define LC_OK              0  // No errors occurred.
#define LC_NO_ARGS         1  // LC_flags is a NULL pointer.
#define LC_MALLOC_ERR      2  // malloc() returned a NULL pointer.

#define LC_BAD_FLAG        3  // A malformed flag was supplied to the program.
#define LC_VAR_RESET       4  // A variable was set twice on the command line.
#define LC_NO_VAL          5  // No value was supplied to a flag that sets a variable.
#define LC_BAD_VAL         6  // A malformed value was supplied.
#define LC_LESS_VALS       7  // Fewer values were supplied than the flag accepts.
#define LC_MORE_VALS       8  // More values were supplied than the flag accepts.
#define LC_FUNC_ERR        9  // A user-defined function returned an error.

#define LC_BAD_VAR_TYPE    10 // The specified flag var_type is invalid.
#define LC_NULL_FORMAT_STR 11 // A NULL pointer was was given for sscanf.
\end{minted}

\monoc{LC_NO_ARGS} and \monoc{LC_BAD_FLAG} mean that we caught non-fatal errors within the program using libClame, while \monoc{LC_MALLOC_ERR} will be passed on a failure to allocate memory. (This could be a result of memory scarcity, but is probably a result of something going very wrong in libc.) Every other return value is a type of user error at the command-prompt.

You can get the constant name given the numeric value using the \monoc{what()} function of the exception type. If necessary, the behaviour of the \monoc{error} integer and the \monoc{what()} function is identical to the return value of \monoc{LC_read()} and \monoc{LC_strerror} respectively.

\subsubsection{Flagless Arguments}

These variables are set by the \monoc{libCLame::read()} function when it encounters arguments supplied to the program that were not preceded by a flag.

\begin{minted}{c++}
extern std::vector<std::string> flagless_args;
\end{minted}

\subsubsection{Program Name}

We also get the program name out of \monoc{argv[0]}.

\begin{minted}{c++}
extern std::string prog_name;
\end{minted}
